\documentclass[journal]{IEEEtran}
\IEEEoverridecommandlockouts
\usepackage{graphics}
\usepackage{rotating}
\usepackage{epsfig}
\usepackage{amsmath}
\usepackage{amssymb}
\usepackage[spanish, es-tabla]{babel}
\usepackage{cite}
\usepackage{hyperref}
\usepackage{float}
\usepackage{csvsimple}
\usepackage[justification=centering]{caption}
\usepackage{atbegshi} % erase first blank page
\AtBeginDocument{\AtBeginShipoutNext{\AtBeginShipoutDiscard}}

\title{\LARGE \bf Análisis de Calidad del Aire en ZMM}

\author{Juan Pablo Echeagaray González \\
Análisis de Métodos Multivariados en Ciencias de Datos \\
MA2003B.301 \\
Dr. José Armando Albert Huerta \\
Dra. Rubí Isela Gutiérrez López \\
7 de noviembre del 2022}% <-this % stops

\begin{document}

    \thanks{Juan Pablo Echeagaray González pertence al Tec de Monterrey Campus Monterrey, N.L. C.P. 64849, Mexico {\tt\small}}

    \maketitle

    \thispagestyle{empty}
    \pagestyle{empty}

    \begin{abstract}
        % Una empresa de e-commerce ha solicitado que alumnos del Tecnológico de Monterrey diseñen un programa de rutas para la entrega de los productos adquiridos por sus clientes en su tienda \emph{online}. Con las bases de datos recibidas se formuló un modelo de CVRP, este modelo fue embebido y optimizado para minimizar las distancias recorridas por los camiones de entrega. El producto final despliega un reporte de las rutas así como una visualización de estas.
    \end{abstract}

    \begin{IEEEkeywords}
        Reducción de dimensionalidad, Pronósticos
    \end{IEEEkeywords}

    \section{Introducción}

        % Descripción de la problemática general y de las metodologías empleadas
        % Estructura general del reporte

    \section{Contexto General}

        % Mencionar las normativas vigentes que regulan la emisión de los contaminantes

    \section{Objeto de estudio}

        % Preguntas de investigación ?

    \section{Delimitación del objeto de estudio}

    \section{Justificación}

        % Justificar el proyecto en general, tal vez ir introduciendo low-key los objetivos del proyecto ?
        % Utilizar info de los efectos de los contaminantes en el cuerpo humano, seres vivos y el entorno

    \section{Marco Teórico}

        % Mencionar los modelos de pronóstico conocidos para la predicción del índice de calidad del aire
        % Hablar de técnicas de reducción de dimensionalidad

    \section{Introducción al cliente}

        % Quien es SIMA?
        % Tareas principales?

    \section{Objetivos}

        % Especificar que el tiempo de desarrollo de este proyecto (en general) estará acotado a una ventana de tiempo de 5 semanas
        % 1. [Justificación] Facilitar que las personas tengan acceso a información en tiempo real de la calidad del aire para generar una concientización de las actividades que la persona como individuo realiza, así como los efectos de los procesos industriales que han elevado la calidad de vida de hoy
        % 2. Proveer de una herramienta de monitoreo especializada en la que un analista pueda checar fácilmente el comportamiento de los contaminantes más relevantes especificados por las normas tal tal tal
        % 3. Entrenamiento de un modelo predictivo que pueda pronosticar el índice de calidad del aire de cada una de las estaciones de SIMA a un horizonte de tiempo de 7 días con una predicción de al menos un 90% en el conjunto de datos de prueba
        % 4. Analizar la posibilidad de implementar una reducción de la dimensionalidad de la base de datos en búsqueda de economizar el proceso de monitoreo de las estaciones así como minimizar el tamaño de las bases de datos generadas

    \section{Metodología}

    \section{Propuesta metodológica}

        \subsection{Entendimiento del negocio}

        \subsection{Análisis y preparación de datos}

        \subsection{Modelado}


        \subsection{Despliegue y evaluación}


    \section{Objetivos de Desarrollo Sostenible}


    \section{Recursos utilizados}


    \section{Resultados}

    \section{Conclusiones}


    \appendices

    \section{Código fuente}\label{code:repo}

        El código generado puede ser consultado \href{https://github.com/JuanEcheagaray75/air-quality-analysis}{aquí}.

    \bibliographystyle{IEEEtran}
    \bibliography{references.bib}

    % Resumen de la revisión de bibliografía
    % Descripción del problema específico (preguntas de investigación) -> Objeto de Estudio
    % Objetivos -> Objetivos
    % Justificación -> Justificación
    %{LIGADOS - Propia Sección
        % Definir en qué sección del reporte va esto
        % Mercado potencial (descripción general)
        % Identificación de clientes/consumidor y usuarios
    %}
    % Descripción de las fuentes de información (datos) -> Propuesta metodológica/Análisis y preparación de datos
    % Selección del modelo (explicar un poco sobre cuales son modelos posibles. Recuerden que estas etapas las podrán modificar más adelante) ->
    % Descripción de la solución (alcance) -> Delimitación del objeto de estudio
    % Propuesta de valor
    % Nombre detallado del proyecto
    % Nombre corto o comercial del proyecto.
    % Impacto social principal -> Justificación
    % Impacto hacia los Objetivos de Desarrollo Sostenible -> ODS
    % Línea de tiempo -> ???
    % Fuentes bibliográficas y de datos

\end{document}