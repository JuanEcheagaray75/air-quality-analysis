\documentclass[10pt]{article}
% Paquetes
\usepackage[utf8]{inputenc}
\usepackage[spanish]{babel}
\usepackage[]{amsthm}
\usepackage{amsmath}
\usepackage[]{amssymb}
\usepackage{graphicx}
\usepackage{wrapfig}
\usepackage[letterpaper, top=0.78in, bottom=0.78in, left=0.98in, right=0.98in]{geometry}
\usepackage[hidelinks]{hyperref}
\usepackage{csvsimple}
\usepackage{pdflscape}
\usepackage{tabularx}
\usepackage{listings}
\usepackage{pdfpages}
\lstset{
    basicstyle=\small\ttfamily,
    columns=flexible,
    breaklines=true}
\renewcommand{\lstlistingname}{Código}
\decimalpoint

% Comandos personalizados
\renewcommand{\baselinestretch}{1.5}
% 20 mm superior e inferior <=> 0.78in
% 25 mm derecha e izquierda <=> 0.98in


\newcolumntype{b}{X}
\newcolumntype{s}{>{\hsize=.5\hsize}X}


\begin{document}
    \begin{titlepage}
        \begin{center}
            % School logo
            \begin{figure}
                \centering
                \includegraphics[scale=0.13]{img/logo_itesm.png}\\ % Logo de la institución
            \end{figure}
            \vspace{5cm}
            % School data
            \LARGE{Instituto Tecnológico y de Estudios Superiores de Monterrey}\\
            \vspace{1cm}
            \large Escuela de Ingeniería y Ciencias \\
            \vspace{0.2cm}
            \large Ingeniería en Ciencias de Datos y Matemáticas \\
            \vspace{0.2cm}
            \large Aplicación de Métodos Multivariados en Ciencias de Datos \\
            \vspace{1cm}
            \textbf{La Calidad del Aire}\\ % Nombre de la tarea
            \vspace{0.7cm}
            % Tabla de integrantes del equipo
            \begin{table}[h!]
                \centering
                \begin{tabular}{ ||c|c|| }
                    \hline
                    Nombre & Matrícula \\
                    \hline
                    Ricardo de Jesús Balam Ek & A00831262 \\
                    \hline
                    Juan Pablo Echeagaray González & A00830646 \\
                    \hline
                    Verónica Victoria García De la Fuente & A00830383 \\
                    \hline
                    Emily Rebeca Méndez Cruz & A00830768 \\
                    \hline
                    Eugenio Santisteban Zolezzi & A01720932 \\
                    \hline
                \end{tabular}
            \end{table}
            \vspace{0.7cm}
            \large Dr. José Armando Albert Huerta \\ % Nombre del profesor 1
            \vspace{0.2cm}
            \large Dra. Rubí Isela Gutiérrez López \\ % Nombre del profesor 2
            \vspace{0.2cm}
            \large Socio Formador: SIMA \\
            \vspace{0.2cm}
            \large Monterrey, Nuevo León \\
            \vspace{0.2cm}
            \large 8 de noviembre del 2022 \\
            \vspace{1cm}
        \end{center}
    \end{titlepage}

    \tableofcontents
    \clearpage

    \section{Introducción}

        Realiza una introducción a tu proyecto basándote en la revisión bibliográfica que hiciste y que aborde estas preguntas y otras que se te ocurran con respecto al tema:

    \begin{itemize}
        \item \textbf{¿Por qué es importante para la ciudadanía el análisis de la calidad del aire?}

        Las actividades diarias de los ciudadanos generan una gran cantidad de sustancias que modifican la composición natural del aire que respiramos. Esta deterioro en la calidad del aire presenta un efecto negativo en la salud humana y del medio ambiente, provocando un aumento de enfermedades respiratorias y cardíacas \cite{sirae}.

        Una manera de proteger la salud de la población por medio del monitoreo y la difusión continua del estado de la calidad del aire; conocer la calidad del aire permite a los ciudadanos saber si es conveniente para ellos realizar actividades en exteriores o tomar medidas para buscar respirar un aire más puro en interiores \cite{sirae}.

        \item \textbf{¿Cuáles son los aspectos que se consideran cuando se analiza la calidad del aire?}
        % Mention the features contained in our database? Or fully talk about all the features in the provided dictionary?

        \item \textbf{¿Cuáles son las clasificaciones internacionales para los contaminantes más importantes? ¿Cuáles son los contaminantes más dañinos para la salud y por qué?}

                De los contaminantes más prominentes en la atmósfera incluyen el Monóxido de Azufre, el Dióxido de Nitrógeno, el Dióxido de Carbono, el Ozono y Partículas Totales en Suspensión \cite{aquae}. Estas partículas pueden provocar una variedad de enfermedades como infecciones respiratorias y cáncer de pulmón. La Organización Panamericana de la Salud junto con la Organización Nacional de la Salud mencionan que estos contaminantes es el problema ambiental más riesgoso en el continente Americano \cite{ops} y que más de 150 millones de habitantes en América Latina viven en lugares con una calidad del aire peligrosa.

        \item \textbf{¿Cómo medir la calidad del aire? ¿qué variables intervendrían en la mejora de la calidad del aire? Conoce el significado de las mediciones.}

            El índice de la calidad del aire (ICA) es la principal manera para medir que tan dañina o saludable es el aire que respiramos en el exterior \cite{aquae}. Esta métrica tiene dominio de 0 a 500, entre más alto sea este valor, más dañino es el aire para la salud. Esta métrica cae también en seis clasificadores según su valor y cada uno se le asigna un color:

            \begin{tabular}{c|c|c}
                 Saludable & Verde & 0-50  \\
                 Moderada & Amarillo & 51-100\\
                 Dañina para algunos & Naranja & 101-150\\
                 Dañina & Rojo & 151-200\\
                 Muy dañina & Morado & 201-250\\
                 Peligrosa & Marrón & 251-300
            \end{tabular}

        \item \textbf{Cuáles son las normas oficiales establecidas de los distintos contaminantes para la protección de la salud.}

        \item \textbf{¿Qué dificultades enfrenta la medición de la calidad del aire?}

        %% Ponerlo mas bonito
        Tener un  buen análisis de cada una de las estaciones de los contaminantes que se emiten.
        Hay datos en el monóxido de carbono tiene un raro comportamiento ya que se debe calibrar los equipos, pero por falta de personal para este trabajo, se comporta de esta manera.
        Algunos datos no se pueden validar porque los equipos se están calibrando o por normas se invalidan, no se puede poner una inputación en los datos inválidos y que mencione que es una aproximación.

        \item \textbf{¿Quién es SIMA? ¿Qué hace? ¿dónde funciona?}

        SIMA es el acrónimo para el Sistema Integral de Monitoreo Ambiental, este tiene la finalidad de contar con información continua y fidedigna de los niveles de contaminación ambiental en el Área Metropolitana de Monterrey. Desde el 20 de noviembre de 1992 la población es informada todos los días del año de la calidad del aire que la población del área metropolitana de Monterrey respira día a día \cite{sima}. Actualmente, opera con 14 estaciones de monitoreo distribuidas en toda la zona metropolitana de Monterrey (ZMM).

    \end{itemize}

    %     Descripción del problema específico (preguntas de investigación)
    %     Objetivos
    %     Justificación
    %     Mercado potencial (descripción general)
    %     Identificación de clientes/consumidor y usuarios
    %     Descripción de la solución (alcance)
    %     Propuesta de valor
    %     Nombre detallado del proyecto
    %     Nombre corto o comercial del proyecto.
    %     Impacto social principal
    %     Impacto hacia los Objetivos de Desarrollo Sostenible

    %     Resumen de la revisión de bibliografía
    %     Descripción de las fuentes de información (datos)
    %     Selección del modelo (explicar un poco sobre cuales son modelos posibles. Recuerden que estas etapas las podrán modificar más adelante)
    %     Línea de tiempo
    %     Fuentes bibliográficas y de datos

    \clearpage
    \bibliographystyle{IEEEtran}
    \bibliography{references.bib}

\end{document}
